%%
\chapter{Conclusions and Outlook} \label{ch:summary}

\section{Conclusions}

Aerosols have been recognized as major components of the Earth's climate 
system, influencing the radiative budget, clouds, and precipitation processes.
An accurate assessment of these effects requires realistic representation of
the aerosol loading and distribution across the globe, as well as the
characteristics of the aerosol particle size distribution and 
chemical composition (refractive index). 

The objective of this thesis is to
contribute an improved research algorithm retrieving aerosol microphysical
properties from AERONET measurements of light radiance and polarization, with
emphasis on elucidating the potentially important role of polarization
measurements. As outlined in section \ref{sec:objective}, specific 
investigations towards this research goal are: the development of an retrieval
algorithm that integrates a rigorous radiative transfer model 
(Chapter \ref{ch:model}) and statistical optimized inversion
(Chapter \ref{ch:algorithm}), the examination
of potential aerosol information contained in the AERONET polarizaton
measurements (Chapter \ref{ch:info}), and the application of our designed
inversion algorithm to the real AERONET measurements (Chapter \ref{ch:case}).
Below, I briefly summarize the contributions of this thesis by these three
investigations.
 
\subsection{UNL-VRTM and new AERONET inversion algorithm}

The UNified Linearized Vector Radiative Transfer Model, or UNL-VRTM integrates
the linearized codes computing vector radiative transfer (VLIDORT) and
scattering of spherical (LMIE) and non-spherical (LTMATRIX) particles, 
hyper-spectral treatment for air molecular scattering and gaseous absorption,
and models describing bidirectional surface reflectance and polarization
(BRDF/BPDF). As shown in Chapter \ref{ch:model}, the direct coupling of these
components by UNL-VRTM allows it not only able to compute the four Stokes
parameters and degree of linear polarization (DOLP) with high accuracy and high
spectral resolution, but also to simultaneously and analytically generate
sensitivities of these Stokes parameters with respect to aerosol parameters of
both the fine and coarse modes. By inclusion of HITRAN and other molecular
spectroscopy data for atmospheric trace gases, the UNL-VRTM is also able to
perform line-by-line calculation of gas absorption, thus providing another
opportunity for the future study of the effect of absorbing gases (such as
\ce{SO2}, \ce{NO2}, \ce{O3}, and water vapor) on the aerosol retrieval. 
Although the UNL-VRTM is used to simulate the AERONET measurements in
this work, the module-based structure of UNL-VRTM allows a broad application
to the remote sensing observations from other platforms.

In Chapter \ref{ch:algorithm}, I have presented a new algorithm to 
retrieve both fine- and coarse-mode aerosol properties from multi-spectral 
and multi-angular solar polarimetric radiation fields measured by AERONET 
including additional spectra of polarization observations. 
The retrieval algorithm uses UNL-VRTM and incorporates the
statistical optimized inversion to retrieve aerosol parameters pertaining to a
bi-lognormal particle size distribution (PSD), including the aerosol volume
concentration, effective radius and variance, and complex indices of
refraction. While the new algorithm has heritage from the existing AERONET
inversion algorithm in using multiple a priori constraints, it is different
from the existing AERONET algorithm in that: (a) a bi-modal lognormal PSD
(instead of 22 size bins) is assumed; (b) the spectral refractive indices are
retrievable for both fine and coarse modes. The mode-separated aerosol
microphysical and optical retrievals can benefit the analysis for aerosol
chemical compositions and climate radiative impacts study of aerosol, 
and most importantly, can thereby facilitate the evaluation of
atmospheric chemistry models and the validation of aerosol products from
satellite sensors with polarization capability.  

\subsection{Potential information contained in AERONET polarization}

In Chapter \ref{ch:info}, I have examined the potential microphysical aerosol
information contained in the AERONET photo-polarimetric observations. The
analysis focused on how the added polarization measurements impact on the
retrieval accuracy the in aerosol particle size distribution, spectral
refractive index, and single scattering albedo. We used the UNL-VRTM to generate
the synthetic AERONET spectral radiance and DOLP, as well as their
sensitivities with respect to these aerosol properties. Then, we quantify the
aerosol infomation content in various observation scenarios in terms of degree
of freedom for signal (DFS) and \textit{a posterior} error.

The results show a remarkable increase in information by adding additional
polarization and/or radiances into the inversion: an overall increase of 2--5
of DFS comparing with radiance-only measurements. Correspondingly, smallest
retrieval errors are found in the added-polarization scenario: 2.3\% (2.9\%) 
for the fine-mode (coarse-mode) aerosol volume concentration, 1.3\% (3.5\%) 
for the effective radius, 7.2\% (12\%) for the effective variance, 
0.005 (0.035) for the real part refractive index, and 0.019 (0.068) for 
the single scattering albedo. These errors represent a reduction from their 
counterparts in the radiance-only scenario of 79\% (57\%), 76\% (49\%), 
69\% (52\%), 66\% (46\%), and 49\% (20\%), respectively. 

We have further investigated those retrieval errors over a variety of 
aerosol loading and fine/coarse-mode prevalence (section \ref{sec:infosensi}),
which indicates that the combined use of radiance
and polarizatuon observations can yield the retrieval of refractive index 
and single scattering albedo for both fine and coarse aerosol modes, 
when AOD at 440 nm is larger than 0.2 and
870/1020-nm Ångström exponent ranges between 0.7 and 1.6. 

\subsection{Application to real retrieval}

In Chapter \ref{ch:case}, we have applied our new AERONET inversion algorithm 
to a suite of real cases over Beijing\_RADI site. We 
found that our retrievals are overall consistent with AERONET operational
inversions, but can offer mode-resolved refractive index and SSA with
acceptable accuracy for the aerosol composed by spherical particles. Along with
the retrieval using both radiance and polarization, we also performed
radiance-only retrieval to demonstrate the improvements by adding polarization
in the inversion. Contrast analysis indicates that with polarization, retrieval
error can be reduced by over 50\% in PSD parameters, 10--30\% in the refractive
index, and 10--40\% in SSA, which is consistent with theoretical analysis
presented in Chapter \ref{ch:info}.

\section{Outlook and Future Work}

The promising results in this study are obtained from the initial development
and preliminary applications of a new algorithm targeted for the retrieval of
aerosol properties from new-generation AERONET measurements. Future
developments will include, but not be limited to, the treatment of
non-spherical large aerosol particles like mineral dust, and the consideration
of tri-modal aerosols for special situations. Another interesting research 
topic is to investigate the chemical composition from the multi-spectral and
multi-angular photo-polarimetric measurements. Below, I list particularly
promising directions for future investigations.

\begin{enumerate}
\item Implement the consideration of non-spherical dust. Conclusions of
of this work are based on consideration of spherical aerosol particles.
However, the studies by \citet{Dubovik06} and \citet{Deuze93, Deuze01}
revealed serious limitation of polarimetric retrieval
of the properties for coarse, especially non-spherical aerosols.
Therefore, treatment of non-spherical optical scattering properties is
necessary to improve our understanding on how polarization can benefit
the retrieval for dust aerosols.  
\item While the bi-lognormal PSD can well represent the aerosol size spectrum 
in most cases, future research efforts  will include the implementation of 
tri-modal aerosol mixtures in situations of cloud formation \citep{Eck12} 
or volcanic aerosols \citep{Eck10}.
\item Explore the potential use of multi-spectral and multi-angular 
photo-polarimetric measurements for the retrieval of aerosol chemical 
composition. This will benifit the source identification of species-specified
aerosols.
\item Last but not least, extensive retrievals for a longer period are
on-going over sites where CE318-DP SunPhotometer instruments 
have been installed (i.e., Beijing\_RADI and Lille). Such long-term
retrievals will provide more robust resources for studying the aerosol 
climatology. 
\end{enumerate}


