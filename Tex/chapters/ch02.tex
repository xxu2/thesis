%%
\chapter{Model Developments} \label{ch:model}

\section{Introduction}

The radiation fields---radiance and the state of polarization---measured by the
AERONET SunPhotometer are the outcome of the interations of solar
radiation with various physical processes including the absorption and
scattering by atmospheric molecules, aerosols and clouds, as well the
reflection and absorption by underlying surface. 
The radiance and polarization of light at any wavelength can be represented by
a Stokes column vector $\mathbf{I}$ having four elements \citep{Hansen74}:
\begin{equation}
\mathbf{I} = [I,Q,U,V]^T,
\end{equation}
where $I$ is the total intensity (or radiance), $Q$ and $U$ describe the state of
linear polarization, $V$ describes the state of circular polarization, and $T$
indicates a transposed matrix. It should be noted that all radiation fields and
optical parameters used in this paper are functions of the light wavelength
$\lambda$. For simplicity, however, we omit $\lambda$ in all formulas. 
The degree of linear polarization ($\dolp$) is defined by
\begin{equation}
\dolp = \frac{\sqrt{Q^2+U^2}}{I}.
\end{equation}

In the solar principal plane, $U$ is negligibly small and the above formula
becomes $\dolp=-Q/I$. Let $\mathbf{I}_0=[I_0,0,0,0]^T$ denote the Stokes vector 
for incident Solar radiation at the top of the atmosphere (TOA) from 
the direction ($\theta_0$, $\phi_0$), where $\theta_0$ and $\phi_0$
are the incident solar zenith and azimuth angles, respectively.
For a plane-parallel atmosphere bounded below by a reflective surface, the
vector radiative transfer equation in the medium for the specific intensity
column vector $\mathbf{I}$ of light propagating in the viewing direction 
($\theta$, $\phi$) can be written \citep{Hovenier04, Mishchenko02}:
\begin{align}
\mu \frac{\partial \mathbf{I}(\tau,\mu,\phi)}{\partial \tau} &=
    \mathbf{I}(\tau,\mu,\phi) - \mathbf{J}(\tau,\mu,\phi; \mu_0, \phi_0)
    \label{eq:rte} \\
\begin{split}
\mathbf{J}(\tau,\mu,\phi; \mu_0, \phi_0) &= 
     \frac{\omega}{4\pi}\int_{-1}^{1}\int_{0}^{2\pi} 
     \mathbf{P}(\tau,\mu,\mu_0,\phi-\phi_0) 
     \mathbf{I}(\tau,\mu_0,\phi_0)\text{d}\phi_0\text{d}\mu_0  \\
     & + \frac{\omega}{4\pi}\mathbf{P}(\tau,\mu,\mu_0,\phi-\phi_0)
     \mathbf{I}_0 \exp(-\tau/\mu_0)
\end{split}
\end{align}
Here, $\tau$ is the extinction optical depth measured from TOA, $\mu$ and
$\mu_0$ are cosines of $\theta$ and $\theta_0$, respectively, $\omega$ is the
SSA and $\mathbf{P}$ is the phase matrix. The first term in equation
\eqref{eq:rte} represents multiple scattering contributions, while
the second indicates scattered light from the direct solar beam. 

Parameters required to solve the above radiative transfer equation are $\tau$,
$\omega$, and $\mathbf{P}(\Theta)$ for the atmosphere, and the reflectance
matrix $\mathbf{R}_\text{s}(\tau,\mu,\phi; \mu_0, \phi_0)$ of 
the underlying surface. Considering a cloud-free atmosphere, the solar
radiation is attenuated by molecular scattering, gaseous absorption, and
aerosol scattering and absorption. For a given layer, we have
\begin{align}
\tau   &= \taua + \taur + \taug \\
\omega &= \frac{\taua\assa + \taur}{\tau} \\
\mathbf{P}(\Theta) &=\mathbf{P}_\text{A}(\Theta)
                     \frac{\taua\assa}{\taua\assa+\taur} 
                    + \mathbf{P}_\text{R}(\Theta)
                    \frac{\taur}{\taua\assa+\taur}
\end{align}
where $\taua$, $\taur$, and $\taug$ are optical depth, respectively, by 
aerosol extinction, Rayleigh scattering of air density fluctuations, 
and gaseous absorption. $\assa$ is the SSA of aerosol,  and
$\mathbf{P}_\text{A}(\Theta)$ and $\mathbf{P}_\text{R}(\Theta)$ are, 
respectively, the aerosol and Rayleigh phase matrices as functions of the 
scattering angle $\Theta$. The forward
modeling of radiance/polarization measurements thus requires knowledge of
single scattering properties for aerosols and air density fluctuations,
absorption of trace gases, and reflectance/polarization by surface. 

\section{The UNL-VRTM}

\subsection{Surface representations}

\subsection{Molecular scattering and absorption}

\subsection{Aerosol single scattering}

\subsection{Radiative transfer}

\subsection{Capability of calculating Jacobians}

\section{Model Benchmarking and Verifications}
