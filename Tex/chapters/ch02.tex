%%
\chapter{Model Developments} \label{ch:model}

\section{Introduction}

The radiation fields---radiance and the state of polarization---measured by the
AERONET SunPhotometer are the outcome of solar radiation interacting
with various physical processes including the absorption and
scattering by atmospheric molecules, aerosols and clouds, as well the
reflection and absorption by underlying surface. 
The radiance and polarization of light at any wavelength can be represented by
a Stokes column vector $\mathbf{I}$ having four elements \citep{Hansen74}:
\begin{equation}
\mathbf{I} = [I,Q,U,V]^T,
\end{equation}
where $I$ is the total intensity (or radiance), $Q$ and $U$ describe the state of
linear polarization, $V$ describes the state of circular polarization, and $T$
indicates a transposed matrix. It should be noted that all radiation fields and
optical parameters used in this paper are functions of the light wavelength
$\lambda$. For simplicity, however, we omit $\lambda$ in all formulas. 
The degree of linear polarization ($\dolp$) is defined by
\begin{equation}
\dolp = \frac{\sqrt{Q^2+U^2}}{I}. \label{eq:dolp}
\end{equation}

In the solar principal plane, $U$ is negligibly small and the above formula
becomes $\dolp=-Q/I$. Let $\mathbf{I}_0=[I_0,0,0,0]^T$ denote the Stokes vector 
for incident Solar radiation at the top of the atmosphere (TOA) from 
the direction ($\theta_0$, $\phi_0$), where $\theta_0$ and $\phi_0$
are the incident solar zenith and azimuth angles, respectively.
For a plane-parallel atmosphere bounded below by a reflective surface, the
vector radiative transfer equation in the medium for the specific intensity
column vector $\mathbf{I}$ of light propagating in the viewing direction 
($\theta$, $\phi$) can be written \citep{Hovenier04, Mishchenko02}:
\begin{align}
\mu \frac{\partial \mathbf{I}(\tau,\mu,\phi)}{\partial \tau} &=
    \mathbf{I}(\tau,\mu,\phi) - \mathbf{J}(\tau,\mu,\phi; \mu_0, \phi_0)
    \label{eq:rte} \\
\begin{split}
\mathbf{J}(\tau,\mu,\phi; \mu_0, \phi_0) &= 
     \frac{\omega}{4\pi}\int_{-1}^{1}\int_{0}^{2\pi} 
     \mathbf{P}(\tau,\mu,\mu_0,\phi-\phi_0) 
     \mathbf{I}(\tau,\mu_0,\phi_0)\text{d}\phi_0\text{d}\mu_0  \\
     & + \frac{\omega}{4\pi}\mathbf{P}(\tau,\mu,\mu_0,\phi-\phi_0)
     \mathbf{I}_0 \exp(-\tau/\mu_0)
\end{split}
\end{align}
Here, $\tau$ is the extinction optical depth measured from TOA, $\mu$ and
$\mu_0$ are cosines of $\theta$ and $\theta_0$, respectively, $\omega$ is the
SSA and $\mathbf{P}$ is the phase matrix. The first term in equation
\eqref{eq:rte} represents multiple scattering contributions, while
the second indicates scattered light from the direct solar beam. 

Parameters required to solve the above radiative transfer equation are $\tau$,
$\omega$, and $\mathbf{P}(\Theta)$ for the atmosphere, and the reflectance
matrix $\mathbf{R}_\text{s}(\tau,\mu,\phi; \mu_0, \phi_0)$ of 
the underlying surface. Considering a cloud-free atmosphere, the solar
radiation is attenuated by molecular scattering, gaseous absorption, and
aerosol scattering and absorption. For a given layer, we have
\begin{align}
\tau   &= \taua + \taur + \taug \\
\omega &= \frac{\taua\assa + \taur}{\tau} \\
\mathbf{P}(\Theta) &=\mathbf{P}_\text{A}(\Theta)
                     \frac{\taua\assa}{\taua\assa+\taur} 
                    + \mathbf{P}_\text{R}(\Theta)
                    \frac{\taur}{\taua\assa+\taur}
\end{align}
where $\taua$, $\taur$, and $\taug$ are optical depth, respectively, by 
aerosol extinction, Rayleigh scattering of air density fluctuations, 
and gaseous absorption. $\assa$ is the SSA of aerosol,  and
$\mathbf{P}_\text{A}(\Theta)$ and $\mathbf{P}_\text{R}(\Theta)$ are, 
respectively, the aerosol and Rayleigh phase matrices as functions of the 
scattering angle $\Theta$. Therefore, the forward modeling development 
thus requires the computation of single scattering properties for aerosols and
air density fluctuations, rigorous treatment for absorption of trace gases, 
accuracte representation of reflectance/polarization by surface, an the
realistic simulation of polarimetric radiative transfer. 

In this regard, we have developed the UNified Linearized Vector
Radiative Transfer Model, or UNL-VRTM, specifically for simulation,
analysis, and inversion of the photo-polarimetric measurements.
Components of the UNL-VRTM are described in section \ref{sec:unlvrtm},
and the model benchmarking and verification are presented in section
\ref{sec:rtmverify}.

\section{The UNL-VRTM} \label{sec:unlvrtm}

As shown in Figure \ref{fig:unlvrtm}, the UNL-VRTM comprises 6
modules; they are 
\begin{enumerate}
\item A module computing Rayleigh scattering (section
\ref{subsec:rayleigh});
\item A module that deal with gaseous absorption (section \ref{subsec:rayleigh});
\item A linearized Mie scattering code (section \ref{subsec:mie});
\item A linearized T-matrix electromagnetic scattering code (section
\ref{subsec:mie});
\item A surface model computing various bidirectional
reflectance/polarization functions (BRDF/BPDF) (section
\ref{subsec:surface});
\item A vector linearized radiative transfer model---VLIDORT (section
\ref{subsec:vlidort}). 
\end{enumerate}
These modules are integrated for the forward calculation of
aerosol single scattering, gas absorption, and vector radiative transfer
hereafter, and thus they together constitute the UNified Linearized
Radiative Transfer Model, UNL-VRTM. 

\begin{figure}[t]
  \centering
  \includegraphics[width={0.95\textwidth}]{figures/unlvrtm.pdf}
  \caption{Flowchart of the UNL-VRTM. See text for detail.}
  \label{fig:unlvrtm}
\end{figure}

Inputs for the UNL-VRTM are profiles of atmospheric properties and
constituents (temperature, pressure, aerosol mass concentration or layer
AOD, water vapor amount and other trace gas volume mixing ratio
profiles), the surface properties, as well as the aerosol parameters 
(such as PSD parameters and refractive index) themselves. 
Bearing in mind the lack of
sensitivity in passive remote sensing for the retrieval of vertical
profiles of aerosol properties, the UNL-VRTM as it stands now is only
designed to deliver radiative calculations for a maximum of two sets of
aerosol single scattering properties (e.g., aerosol PSD,
refractive index, and particle shape), typically with one fine-mode and
one coarse-mode aerosol. Other inputs for model include spectral and
geometrical definitions that characterizing specification of an
observing sensor. 

Outputs of the model include the Stokes vector ($\mathbf{I}$) at
user-defined spectral wavelengths and desired atmospheric
levels for both upwelling and downwelling radiation, from which the
light radiance and degree of polarization can be derived. Outputs also
include analytical Jacobians of $\mathbf{I}$ with respect to all aerosol
particle parameters (PSD, refractive index, vertical profile), Rayleigh
scattering optical depth, optical depth of all trace gases, and
parameters describing surface optical property. A detail description of
the UNL-VRTM's Jacobian capability is presented in section
\ref{subsec:jacobian}. 

\subsection{Molecular scattering and absorption} \label{subsec:rayleigh}

The Rayleigh scattering optical depth at certain wavelength in any 
atmospheric layer ($\taur$) is computed by
\begin{equation}
\taur = N_\text{air}\sigma_\text{R} 
\end{equation}
where $N_\text{air}$ is air molecular number density of that layer
(molec cm$^{-2}$), and $\sigma_\text{R}$ is the Rayleigh scattering 
cross-section (cm$^2$ molec$^{-1}$) computed following
\citet{Bodhaine99}. The Rayleigh phase matrix,
$\mathbf{P}_\text{R}(\Theta)$, depends upon molecular 
anisotropy through the depolarization factor, also computed from the same 
source. \citet{Bodhaine99} computes the wavelength-dependent Rayleigh
scattering cross-section as a function of mixing ratios for \ce{N2},
\ce{O2}, \ce{H2O}, and \ce{CO2}. The phase matrix for Rayleigh scattering
follows \citet{Hansen74}; we use the set of spherical-function expansion
 coefficients for the phase matrix as supplied for VLIDORT
\citep{Spurr06}.

Calculation of the absorption optical depth ($\taug$) at
any atmospheric layer for $K$ different trace gases follows
\begin{equation}
\taug = \sum_{i=1}^K N_{\text{gas,}i}\sigma_{\text{A,}i}(T,P) 
\end{equation}
where $N_{\text{gas,}i}$ is the number density of \textit{i}th gas
within that layer, and $\sigma_{\text{A,}i}$ is the corresponding
absorption cross-section, a function of temperature and pressure. 
Our model accounts for absorptions by a total number of
22 trace gases: \ce{H2O}, \ce{CO2}, \ce{O3}, \ce{N2O}, \ce{CO},
\ce{CH4}, \ce{O2}, \ce{NO}, \ce{SO2}, \ce{NO2}, \ce{NH3}, \ce{HNO3},
\ce{OH}, \ce{HF}, \ce{KCl}, \ce{HBr}, \ce{HI}, \ce{ClO}, \ce{OCS},
\ce{H2CO}, \ce{HOCl}, and \ce{N2}.
The determination of $\sigma_\text{A}$  
utilizes a UV-to-visible cross-section library and the line-spectroscopic
absorption parameters archived in the HITRAN database \citep{Orphal03,
Rothman09}. The cross-section library compiles the extinction
cross-section for \ce{O3}, \ce{NO2}, \ce{SO2}, and \ce{O2-O2} in the UV
and/or visible spectral regions. Meanwhile, line-spectroscopic 
absorption databased are used to simulate the pressure- and
temperature-dependent extinction cross-section with line-by-line (LBL) 
approach \citep{Liou02,Rothman09} by accumulating each individual 
absorption line. Doppler broadening is calculated from the molecular mass
and the temperature, and Doppler and Lorentz broadening are included in the
Voigt calculation.

Particular to work, we only consider the most influential 
trace species for the AERONET spectral bands: \ce{H2O} (vapor), \ce{O3}, 
and \ce{NO2}. In our algorithm (section \ref{ch:algorithm}), the columnar
amounts of \ce{O3} and \ce{NO2} are dynamically adjusted with retrievals from the
Ozone Monitoring Instrument (OMI) \citep{Levelt06} on board the
AURA satellite. We apply the columnar water vapor amount retrieved from
the 940-nm radiances measured by the AERONET SunPhotometer
\cite{Halthore97}.

\subsection{Aerosol single scattering} \label{subsec:mie}

Aerosol single scattering properties necessary to the radiative
transfer calculation include aerosol optical depth ($\taua$)
($\qext$), SSA ($\assa$), and scattering phase matrix ($\paer(\Theta)$).
The calculation of these parameters is made with a Linearized Mie (LMIE)
scattering electromagnetic code for spherical particles and a Linearized
T-matrix (LTMATRIX) scattering code for non-spherical convex and axially
symmetric particles \citep{Spurr12}. The LMIE code originates from the Mie code
of \citet{deRooij84}, and the LTMATRIX code originates from the T-Matrix
code developed by \citet{Mishchenko96, Mishchenko98}; both include
linearization capabaility developed by \citet{Spurr12}.

Common inputs for both codes are the complex refractive index
($\mreal + i\mimag$), and the particle size distribution (PSD) 
parameters for polydisperse scattering. The codes have several options 
to specify the PSD function: two-parameter gamma, two-parameter 
lognormal, three-parameter modified gamma, and four-parameter bi-lognormal. 
In addition, the linearized T-matrix code offers options to characterize the 
shape of non-spherical aerosols (spheroids, cylinders, or Chebyshev particles)
\citep{Spurr12}. For non-spherical particles, the specified size distribution 
is interpreted as the equivalent surface-area sphere in the linearized T-matrix
calculation, regardless of the shape. 

For AERONET inversion alorithm, we assume that the aerosol volume
distribution follows a bi-modal lognormal function \citep[in agreement
wit][]{Schuster06, Waquet09}:
\begin{equation}
\frac{\text{d}V}{\text{d}\ln{r}} = \sum_{i=1}^2 
\frac{V_0^i}{\sqrt{2\pi}\ln{\sg^i}} 
\exp{\left[ -\frac{(\ln{r}-\ln{\rv^i})^2}{2\ln^2{\sg^i}}\right]}
\end{equation}
where $V_0$, $\rv$, and $\sg$ are the total volume concentration, geometric
median radius, and standard deviation, respectively. The superscript $i$
indicates the size mode, and later will be replaced by ‘f’ for fine mode
and ‘c’ for coarse mode. We assume that particle size ranges from 0.01
to 10 $\mu$m for the fine mode and from 0.05 to 20 $\mu$m for
the coarse mode, both covering $>$ 99.9\% of the total volume of an 
idealistic size range (0, $+\infty$). An advantage of the lognormal 
distribution is that standard deviations for the number, area, and 
volume PSD functions are identical, and therefore allowing that the 
median radii for these PSD functions can be converted from one to 
another \citep{Seinfeld06}. The $\reff$ and $\veff$ are related to 
the geometric parameters through:
\begin{align}
\reff &= \rv \exp{\left( -\frac{1}{2} \ln^2{\sg}\right)}, \\
\veff & = \exp{\left(\ln^2{\sg}\right)} - 1.
\end{align}

The LMIE/LTMATRIX code computes the aerosol extinction efficiency
factor $\qext$, single scattering albedo $\assa$, and phase matrix 
$\paer(\Theta)$, as well as Jacobians of these quantities with respect
to input parameters including $\reff$, $\veff$, $\mreal$, and $\mimag$.
The phase matrix and its Jacobians are expressed in terms of the
coefficients $\baer(\Theta)$ for each moment $l$ in terms of the 
generalized spherical function expansions
for each non-zero phase matrix element. Let $\mathbf{A}$ denotes the
vector of aerosol microphysical parameters, 
$\mathbf{A}=[V_0,\reff,\veff,\mreal,\mimag]^T$, and $\mathbf{M}$
the vector of aerosol optical parameters, 
$\mathbf{M}=[\taua,\assa,\baer(\Theta)]^T$, 
where $\taua$ is related to $\qext$ by 
$\taua=\frac{3V_0 \qext}{4\reff}$. The LMIE/LTMATRIX code acts
as an operator that maps vector $\mathbf{A}$ to $\mathbf{M}$. 
The Jacobian matrix of $\mathbf{M}$ with respect to $\mathbf{A}$ calculated by 
means of the linearization feature of the code, and it can be expressed by
$\nabla_\mathbf{A}\mathbf{M}$. 

\subsection{Surface representations} \label{subsec:surface}

VLIDORT has a supplementary module for specification of the surface BRDF
as a linear combination of (up to) three semi-empirical kernel
functions; for details, see \citet{Spurr04}. This supplementary module can also
provide partial derivatives of the BRDF with respect to the kernel
weighting factors or with respect to kernel parameters such as the wind
speed for glitter reflectance. These kernel functions include
Lambertian, Ross-Thick, and Li-Sparse functions \citep{Wanner95, Lucht00}, a
Bi-directional Polarization Distribution Function (BPDF)
\citep{Maignan09}, and an ocean surface model based on the 
Cox-Munk model \citep{Cox54}. In addition, VLIDORT has
an option for using a surface-leaving radiation field, either as a
fluorescence term or as a water-leaving term expressed as a function of
chlorophyll absorption.

Although surface reflectance has in general a low influence on AERONET
down-welling sky radiances and polarization, a state-of-the-art
representation of the surface reflectivity potentially reduces model
uncertainties, especially for measurements taken at low elevation angles
that could be affected by surface diffusion. Here, we utilize the
spectral BRDF parameters from the MODIS surface products that are
operationally reported every 16 days at a 1-km resolution
\citep{Lucht00}. Here we use time-matched MODIS BRDF products 
to reconstruct the bidirectional reflectance over AERONET stations.
The MODIS BRDF product supplies three weighting parameters
($f_\text{iso}$, $f_\text{vol}$, and $f_\text{geo}$) for the first 
7 MODIS bands, respectively, corresponding to three kernel types:
isotropic, Ross-Thick ($K_\text{vol}$), and Li-Sparse ($K_\text{geo}$):
\begin{equation}
\rho_\text{R}(\mu,\phi;\mu_0,\phi_0) = f_\text{iso} + 
f_\text{vol}K_\text{vol}(\mu,\phi;\mu_0,\phi_0) +
f_\text{geo}K_\text{geo}(\mu,\phi;\mu_0,\phi_0)  \label{eq:brdf}
\end{equation}
Expanded expressions for $K_\text{vol}$ and $K_\text{geo}$ appear in
\citet{Wanner95, Lucht00}. 

Studies have shown that the BPDF for land surfaces is generally rather
small and is “spectrally neutral” \citep{Nadal99, Maignan04, Maignan09,
Waquet07, Litvinov11}. 
Most empirical BPDF models are based on Fresnel coefficients of light
reflectance from the surface. Here we have incorporated the
one-parameter model developed by \citet{Maignan09}, which was
derived from analyses of several years of POLDER/PARASOL measurements.
This model describes the polarized reflectance at any viewing geometry
($\mu$, $\phi$) from the given incident geometry ($\mu_0$, $\phi_0$) as:
\begin{equation}
\rho_\text{P}(\mu,\phi;\mu_0,\phi_0) = 
\frac{C_0 \exp(-\tan \theta_\text{h})\exp(-\text{NDVI})}{\mu_0+\mu} 
\mathbf{F}_\text{P}(\theta_\text{h},n_\text{v}) \label{eq:bpdf}
\end{equation}
where $C_0$ is a constant parameter chosen for a certain surface type,
$\theta_\text{h}$ is half of the phase angle of reflectance, $n_\text{v}$ is 
the refractive index of vegetation (1.5 is used), and
$\mathbf{F}_\text{P}$ is the Fresnel reflection matrix. 
We chose a spectrally-independent value for $C_0$ based
on the recommendations by \citet{Maignan09} for relevant surface types. 

The combination of the BRDF and BPDF for land surface follows the
discussion by \citet{Dubovik11}. The surface reflectance matrix 
$\mathbf{R}_\text{s}(\mu,\phi;\mu_0,\phi_0)$
is represented as a sum of diffuse unpolarized
reflectance and specular reflectance; the former is modeled using the
MODIS BRDF in equation \eqref{eq:brdf}, and the latter using the BPDF 
formula in equation \eqref{eq:bpdf}. 

\subsection{Radiative transfer} \label{subsec:vlidort}

The radiative transfer equation \eqref{eq:rte} is solved with the 
Vector Linearized Discrete Ordinate Radiative Transfer (VLIDORT)
model, which is a core part of the UNL-VRTM. VLIDORT, 
developed by \citet{Spurr06}, is a
linearized pseudo-spherical vector discrete ordinate radiative transfer
model for multiple scattering of diffuse radiation in a stratified
multi-layer atmosphere. It computes four elements of the Stokes vector I
for downwelling and upwelling radiation at any desired atmospheric
level. The VLIDORT includes the pseudo-spherical approximation to
calculate solar beam attenuation in a curved medium. It also uses the
delta-M approximation for dealing with sharply peaked forward
scattering. Specifically for the AERONET inversion, we consider 16
discrete ordinate streams in the radiative transfer calculation and
retain 180 terms in the spherical-function expansion of the scattering
matrix to ensure accurate calculation of diffuse radiation. 

Along with the Stokes vector $\mathbf{I}$, VLIDORT also computes the Jacobian
matrix of I with respect to aerosol optical vector $\mathbf{M}$, 
$\nabla_\mathbf{M}\mathbf{I}$. Therefore,
the combination of the VLIDORT and the LMIE/LTMATRIX codes allows for a
direct calculation of the Jacobian matrix of the Stokes vector with
respect to aerosol microphysics, $\mathbf{A}$, by
\begin{equation}
\nabla_\mathbf{A}\mathbf{I} = \nabla_\mathbf{M}\mathbf{I}
\cdot \nabla_\mathbf{A}\mathbf{M}
\end{equation}
Essentially, the above equation can yield the derivatives of the
radiance $I$ and $\dolp$ with respect to any aerosol microphysical parameter,
i.e., $\nabla_\mathbf{A}I$ and $\nabla_\mathbf{A}\dolp$. 
While obtaining $\nabla_\mathbf{A}I$ is straightforward,
$\nabla_\mathbf{A}\dolp$ can be derived from equation \eqref{eq:dolp} 
following:
\begin{equation}
\nabla_\mathbf{A}\dolp = -\frac{\dolp \nabla_\mathbf{A}}{I} +
\frac{Q\nabla_\mathbf{A}{Q}+U\nabla_\mathbf{A}{U}}{I\sqrt{Q^2+U^2}}
\end{equation}

\subsection{Capability of calculating Jacobians} \label{subsec:jacobian}

This section analytically derives the Jacobian of $\mathbf{I}$ with
respect to various aerosol related parameters, including $\taua$,
$\assa$, $\baer$, refractive index, PSD parameters, and aerosol vertical
profile.

\begin{table}[t]
  \centering
  \small
  \caption{Elements of transformation vector for various aerosol single
scattering parameters (composite of fine and coarse mode).}
  \label{tab:jacobian1}
  \begin{tabular}{p{2em} p{2em} p{5em} p{15em} }
    \toprule
       $x$ & $\phi_x$ & $ \varphi_x$ & $\Psi_x^j$ \\
    \midrule
    $\taua$        & $\frac{\taua}{\tau}$ &
$\frac{\taua}{\tau}\left( \frac{\assa}{\omega}-1\right)$ & 
$\left\{ \begin{array}{ll} \frac{\assa\taua}{\omega\tau}\left(
\frac{\baer^j}{\mathbf{B}^j}-1 \right)
&\mbox{for } j<3  \\ \frac{\taur}{\omega\tau} & \mbox{for } j\geq 3 \end{array} \right.$\\  
    $\assa$        & 0 & $\frac{\taua\assa}{\tau\taua\assa+\taur}$ &
Same as above \\
    $B_\text{A}^j$ & 0 & 0 & $\left\{ \begin{array}{ll}
\frac{\assa\taua\baer^j}{\assa\taua\baer^j+\taur\mathbf{B}_\text{R}^j}
&\mbox{for } m=j<3  \\ 1 & \mbox{for } m=j\geq 3 \\ 0 & \mbox{for } m\neq j
\end{array} \right.$ \\
    \bottomrule
  \end{tabular}
\end{table}


\section{Model Benchmarking and Verifications} \label{sec:rtmverify}
