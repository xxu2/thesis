%% You have a maximum of 350, which includes your title, name, etc.
\begin{abstract}
Atmospheric aerosols play an important role in earth climate change by
scattering and absorbing solar and terrestrial radiation, and indirectly
through altering the cloud formation, lifetime, and radiative properties.
However, accurate quantification of these effects is in no small part hindered
by our limited knowledge about the particle size distribution (PSD) and
refractive index, the aerosol microphysical properties essentially pertain to
aerosol optical and cloud-forming properties. The research goal of this thesis
is to obtain these properties of both fine and coarse aerosol modes from the
direct and diffuse polarimetric solar radiation measured by the SunPhotometer
of the Aerosol Robotic Network (AERONET). We achieve so by (1) developing an
inversion algorithm that integrates rigorous radiative transfer model with a
statistical optimization approach, (2) conducting a sensitivity study and error
budgeting exercise to examine the potential value of adding polarization to the
current radiance-only inversion, and (3) performing retrievals using available
AERONET polarimetric measurements. 

The results from theoretical information and error analysis indicate a
remarkable increase in information by adding additional polarization and/or
radiances into the inversion: an overall increase of 2--5 of degree of freedom
for signal comparing with radiance-only measurements. Correspondingly,
retrieval uncertainty can be reduced by 79\% (57\%), 76\% (49\%), 69\% (52\%),
66\% (46\%), and 49\% (20\%) for the fine-mode (coarse-mode) aerosol volume
concentration, the effective radius, the effective variance, the real part of
refractive index, and single scattering albedo, respectively, resulting in
their retrieval errors of 2.3\% (2.9\%), 1.3\% (3.5\%), 7.2\% (12\%), 0.005 
(0.035), and 0.019 (0.068).

In real cases, we demonstrate that our retrievals are overall consistent with
current AERONET operational inversions, but can offer mode-resolved refractive
index and SSA with sufficient accuracy for the aerosol composed by spherical
particles. Along with the retrieval using both radiance and polarization, we
also performed radiance-only retrieval to reveal the improvements by adding
polarization in the inversion. Such contrast analysis indicates that with
polarization, retrieval error can be reduced by over 50\% in PSD parameters,
by 10--30\% in the refractive index, and by 10--40\% in SSA, which is 
consistent with the theoretical results.  

\end{abstract}

%% Optional opyrightpage
%\begin{copyrightpage}
%This file may be distributed and/or modified under the conditions of
%the \LaTeX{} Project Public License, either version 1.3c of this license
%or (at your option) any later version.  The latest version of this
%license is in:
%\begin{center}
%   \url{http://www.latex-project.org/lppl.txt}
%\end{center}
%and version 1.3c or later is part of all distributions of \LaTeX version
%2006/05/20 or later.
%\end{copyrightpage}

%%%%% Optional
%\begin{dedication}
%\begin{center}
%  \vskip.6in
%  To the memory of my grandfather \\
%  \vskip0.2in
%\begin{figure}[h]
%  \centering
%  \includegraphics[width={0.35\textwidth}]{figures/grandpapa.jpg}
%  \caption*{\textbf{Zhaoxiang Xu} \\
%(1933 -- 2014)}
%\end{figure}
%\end{center}
%\end{dedication}

%% Optional
\begin{acknowledgments}
  Acknowledgment to be filled \ldots
\end{acknowledgments}

%% Optional
\begin{grantinfo}
  The NASA Earth and Space Science Fellowship funded this project from September
  2012 to August 2015. I am also grateful to the support
  from NASA's New Investigator Program and Radiation Science Program
  (to Dr. Jun Wang).
\end{grantinfo}
