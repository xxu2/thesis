%% You have a maximum of 350, which includes your title, name, etc.
\begin{abstract}
Atmospheric aerosols play an important role in earth climate by
scattering and absorbing solar and terrestrial radiation, and indirectly
through altering the cloud formation, lifetime, and radiative properties.
However, accurate quantification of these effects is in no small part hindered
by our limited knowledge about the particle size distribution (PSD) and
refractive index, the aerosol microphysical properties essentially pertain to
aerosol optical and cloud-forming properties. The research goal of this thesis
is to obtain the aerosol microphysical properties of both fine and coarse 
modes from the polarimetric solar radiation measured by the SunPhotometer
of the Aerosol Robotic Network (AERONET). We achieve so by (1) developing an
inversion algorithm that integrates rigorous radiative transfer model with a
statistical optimization approach, (2) conducting a sensitivity study and error
budgeting exercise to examine the potential value of adding polarization to the
current radiance-only inversion, and (3) performing retrievals using available
AERONET polarimetric measurements. 

The results from theoretical information and error analysis indicate a
remarkable increase in information by adding additional polarization
into the inversion: an overall increase of 2--5 of degree of freedom
for signal comparing with radiance-only measurements. Correspondingly,
retrieval uncertainty can be reduced by 79\% (57\%), 76\% (49\%), 69\% (52\%),
66\% (46\%), and 49\% (20\%) for the fine-mode (coarse-mode) aerosol volume
concentration, the effective radius, the effective variance, the real part of
refractive index, and single scattering albedo (SSA), respectively, resulting in
their retrieval errors of 2.3\% (2.9\%), 1.3\% (3.5\%), 7.2\% (12\%), 0.005 
(0.035), and 0.019 (0.068).

In real cases, we demonstrate that our retrievals are overall consistent with
current AERONET operational inversions, but can offer mode-resolved refractive
index and SSA with sufficient accuracy for the aerosol composed by spherical
particles. Along with the polarimetric retrieval, we
also performed radiance-only retrieval to reveal the improvements by adding
polarization in the inversion. The comparison analysis indicates that with
polarization, retrieval error can be reduced by over 50\% in PSD parameters,
by 10--30\% in the refractive index, and by 10--40\% in SSA, which is 
consistent with the theoretical results.  

\end{abstract}

%% Optional opyrightpage
%\begin{copyrightpage}
%This file may be distributed and/or modified under the conditions of
%the \LaTeX{} Project Public License, either version 1.3c of this license
%or (at your option) any later version.  The latest version of this
%license is in:
%\begin{center}
%   \url{http://www.latex-project.org/lppl.txt}
%\end{center}
%and version 1.3c or later is part of all distributions of \LaTeX version
%2006/05/20 or later.
%\end{copyrightpage}

%%%%% Optional
\begin{dedication}
\begin{center}
  \vskip.6in
  To \\
  my parents, my wife Dan, and our beloveds boys Aaron and Alex \\
  for their constant support and unconditional love,
  \vskip1em
  and also to \\
  the memory of \\
my grandfather Zhaoxiang Xu.

\end{center}
\end{dedication}

%% Optional
\begin{acknowledgments}
While I am immensely thankful for the help, support, and encouragement
from many people, I am especially grateful to my advisor, Professor
Jun Wang. I have been very fortunate to have an excellent advisor 
to guide me through my PhD. I have learned much from Jun's exemplary 
scientific and personal integrity, his courage to think outside the box, 
and his generosity to promote his students and associates.
From the moment I started working with him, Jun has been inspiring me to
pursue my own ideas and interests and encouraging me to "be proactive"
and "take the ownership of my work". Jun is an advisor who is really
supportive not only to the scientific work but also to the academic
career of his students. Thanks to Jun I have attended dozens of 
conferences to present my research and to get connected to the peers and 
scientists in my research fields. I sincerely appreciate the genuine
support and investment Jun has shown in my continued academic success. 

I would like to thank my thesis committee: Anatoly Gitelson, Daven Henze, 
Steve Hu, and Clinton Rowe. They guided me through all these years and
have been providing constructive criticism on my manuscripts 
and non-stoping help on my academic progress. Daven guided me through 
my learning on the GEOS-Chem adjoint modeling and my
implementing the adjoint modeling of dust emissions. I benefit a lot from
many discussions with Daven to my understanding on the inverse modeling
and optimization, which is the core methodology of this thesis. I had
served as a Teaching Assistant to Clinton for the course of Atmospheric
Thermodynamics, from which I gained valuable teaching experience. 

I would like to acknowledgement all the great help I received from
people in the department. I thank David
Watkins and Tracy Frank, our former and current department chairs, for
their advice, support, and making sure Bessey Hall an enjoyable place to
work. I am also grateful to our secretaries, Tina Gray and Janelle Gerry, 
for logistical support. I thank Deborah Bathke, Adam Houston, 
Robert Oglesby, and Lily Zeng for teaching my classes, which indeed are 
sources and foundations of my research ideas. 
A special acknowledgement goes to people of
Jun's research group: Clint Aegerter, Chase Calkins, Shouguo Ding, Cui Ge, 
Weizhen Hou, David Peterson, Thomas Polivka, Ambrish Sharma, Yi Wang,
Zhifeng Yang, and Yun Yue, who have provided
great advice and help in many ways. I am also thankful to have great
office mates during these years, Limpert and Curtis Walker, making my
stay in the office incredibly enjoyable. 

I would also like to thank outside people who have given me support and 
guidance. I appreciate the generosity of Robert Spurr for his sharing
and guidance of his codes of radiative transfer and particulate
scattering, without which my development of the UNL-VRTM is impossible.
I thank Xiong Liu and Kelly Chance for their help on implementing the HITRAN
capability to UNL-VRTM. I thank Li Li and Zhengqiang Li for providing
measurement data to support my thesis. I am also indebted to Oleg
Dubovik for several discussions that helped me to understand the
numerical inversion theory, to Brent Holben and Michael Mishchenko for
reading my manuscripts and providing valuable comments. Acknowledgement
also goes to Michael Garay, Paul Ginoux, Daniel Jacob, Olga Kalashnikova, 
Feng Xu, and Ping Yang for their genuine help and encouragements. 

Many friends have made my stay here in Lincoln incredibly fun. Thank you
Roger and Gery, Yang Gao and Min, Yanbin and Cuicui, Jinya and Lishan, 
Zhe Yuan, Ruopu and Leiming, Xiaopeng and Yi, and many people I might
forgot to mention. I would also like to our Messiah Lutheran Church and
its community to strengthen my faith. 

Finally, none of this would have been possible without the unconditional
love and support from my parents, my wonderful wife Dan and our beloved boys 
Aaron and Alexander, and my nice mother-in-law Rui. Thank you all!  

\end{acknowledgments}

%% Optional
\begin{grantinfo}
  The NASA Earth and Space Science Fellowship funded this project from
  2012 to 2015. I am also grateful to the support
  from NASA's New Investigator Program and Radiation Science Program
  (to Dr. Jun Wang).
\end{grantinfo}
