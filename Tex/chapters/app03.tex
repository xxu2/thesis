%%% Apendix 03
\chapter{Derivations of Transformation Vector $\pmb{\Pi}$} \label{app:pi}

Let $x$ be an aerosol microphysical parameter. The aerosol extinction and
scattering optical thickness ($\taua$ and $\scaa$), single scattering albedo
($\assa$), and Greek coefficient matrix ($\baer^j$) are functions of $x$. 
However, the gaseous absorption and Rayleigh scattering parameters are 
independent of $x$. This appendix outlines the derivations of equations 
\eqref{eq:trans} and \eqref{eq:pi} and the expressions in
Table \ref{tab:jacobian1} and Table \ref{tab:jacobian2}. 

First, we transform equation \eqref{eq:lopt} as below:
\begingroup
\allowdisplaybreaks
\begin{align}
\phi_x &= \frac{x}{\tau}\frac{\pttau}{\ptx} 
        = \frac{x}{\tau}\frac{\partial(\taug+\taur+\taua)}{\ptx} 
        = \frac{1}{\tau}x\frac{\pttaua}{\ptx} \label{eq:phix1} \\
        \nonumber \\
\varphi_x 
        &= \frac{x}{\omega}\frac{\ptssa}{\ptx} 
         = \frac{x}{\omega}\frac{[(\scaa+\taug)]}{\ptx}
         \nonumber \\
        &= \frac{x}{\omega}\frac{1}{\tau^2}
           \left[ \tau\frac{\partial{(\scaa+\taur)}}{\ptx}
                - (\scaa+\taur)\frac{\pttau}{\ptx}\right]
         \nonumber \\
        &= \frac{x}{\omega\tau}\frac{\ptscaa}{\ptx}
           -(\scaa+\taur)\frac{x}{\omega\tau^2}\frac{\pttaua}{\ptx}
         \nonumber \\
        &= \frac{x}{\scaa+\taur}\frac{\ptscaa}{\ptx}
           -\frac{1}{\tau}\frac{\pttaua}{\ptx}
         \nonumber \\
        &= \frac{x}{\scaa+\taur}\frac{\ptscaa}{\ptx} - \phi_x \label{eq:phix2}\\
         \nonumber \\
\pmb{\Psi}_x^j 
        &= \frac{x}{\Bbf^j}\frac{\ptb^j}{\ptx} 
         = \frac{x}{\Bbf^j}\frac{\partial{[(\taur\bray^j+\scaa\baer^j)/(\scaa+\taur)]}}{\ptx}
         \nonumber \\
        &= \frac{x}{\Bbf^j}\frac{1}{(\scaa+\taur)^2}\left[
           (\scaa+\taur)\frac{\partial(\scaa\baer^j)}{\ptx}
           - (\taur\bray^j+\scaa\baer^j)\frac{\ptscaa}{\ptx}\right]
         \nonumber \\
        &= \frac{x}{\Bbf^j}\frac{1}{\scaa+\taur} \left[
           \frac{\partial(\scaa\baer^j)}{\ptx}-\Bbf^j\frac{\ptscaa}{\ptx}\right]
         \nonumber \\
        &= \frac{1}{(\scaa+\taur)\Bbf^j} \left[ 
           \scaa x\frac{\baer^j}{\ptx}+(\baer^j-\Bbf^j)x\frac{\ptscaa}{\ptx}
           \right] \label{eq:phix3}
\end{align}
\endgroup

These expressions are linear combinations of $\phi_x^\prime$,
$\varphi_x^\prime$, and $\pmb{\Psi}_x^{\prime j}$ (as defined by equation
\eqref{eq:alopt}), where 
\begin{equation}
\left[ \phi_x^\prime, \varphi_x^\prime, \left<
\pmb{\Psi}_x^{\prime j}\right>_{j=1,J}\right]^T = 
\left[x\frac{\pttaua}{\ptx}, x\frac{\ptscaa}{\ptx}, 
      \left<x\frac{\ptbaer^j}{\ptx}\right>_{j=1,J} \right]^T
\end{equation}
We then can write above equations \eqref{eq:phix1}--\eqref{eq:phix3} into
vector formulism (as equation \eqref{eq:trans}:
\begin{equation}
\left[ \phi_x, \varphi_x, \left< \pmb{\Psi}_x^j\right>_{j=1,J}\right]^T
=\pmb{\Pi} \left[ \phi_x^\prime, \varphi_x^\prime, \left<
\pmb{\Psi}_x^{\prime j}\right>_{j=1,J}\right]^T \label{eq:apptrans}
\end{equation}
where $\pmb{\Pi}$ is a matrix comprising the relevant coefficients, 
as noted in equation \eqref{eq:pi}. Equations \eqref{eq:apptrans} and 
\eqref{eq:pi} then act as a universal formulation for preparing
linearized inputs of optical property for VLIDORT. Computation of
$\left[ \phi_x, \varphi_x, \left< \pmb{\Psi}_x^j\right>_{j=1,J}\right]$ 
can then be achieved by the calculation of
$\left[ \phi_x^\prime, \varphi_x^\prime, \left<\pmb{\Psi}_x^{\prime j}\right>_{j=1,J}\right]$ 
for a given parameter x.

Let us first consider the derivation of 
$\left[ \phi_x^\prime, \varphi_x^\prime, \left<\pmb{\Psi}_x^{\prime j}\right>_{j=1,J}\right]$
for certain aerosol optical properties in a given atmospheric layer, 
i.e., $\taua$, $\assa$, and $\betaak$, where $\betaak$ 
indicates one of the elements in the $k$th aerosol scattering Greek
matrix $\baer^k$. 

For $x=\taua$, we have
\begingroup
\allowdisplaybreaks
\begin{align}
\phi_x^\prime &= \taua\frac{\pttaua}{\pttaua}=\taua \label{eq:adjtau} \\
\varphi_x^\prime &= \taua\frac{\ptscaa}{\pttaua}=\taua\assa \\
\pmb{\Psi}_x^{\prime j} &= \taua\frac{\ptbaer^j}{\pttaua} =\pmb{0} 
\end{align}
\endgroup

For $x=\assa$, we have
\begingroup
\allowdisplaybreaks
\begin{align}
\phi_x^\prime &= \assa\frac{\pttaua}{\ptassa}=0 \\
\varphi_x^\prime &= \assa\frac{\ptscaa}{\ptassa}=\assa\taua \\
\pmb{\Psi}_x^{\prime j} &= \assa\frac{\ptbaer^j}{\ptassa} =\pmb{0} 
\end{align}
\endgroup

For $x=\betaak$, we have
\begingroup
\allowdisplaybreaks
\begin{align}
\phi_x^\prime &= \betaak\frac{\pttaua}{\ptbetaak}=0  \\
\varphi_x^\prime &= \betaak\frac{\ptscaa}{\betaak}=0 \\
\pmb{\Psi}_x^{\prime j} &= \betaak\frac{\ptbaer^j}{\betaak} = 
\begin{cases}
 \frac{\scaa\betaak}{\beta^k} & \mbox{if $j=k$} \\
 0 & \mbox{if $j\neq k$}
\end{cases}
\label{eq:adjbeta}
\end{align}
\endgroup

Expressions in Table \ref{tab:jacobian1} are then derived by substituting 
equations \eqref{eq:adjtau}--\eqref{eq:adjbeta} into equation
\eqref{eq:apptrans}.

The UNL-VRTM integrates the VLIDORT with linearized Mie/T-matrix codes, and
this combination allows us to generate Stokes vectors and associated analytical
Jacobians with respect to aerosol microphysical parameters for two aerosol
modes. Thus, we must supply the 
$\left[ \phi_x^\prime, \varphi_x^\prime, \left<\pmb{\Psi}_x^{\prime j}\right>_{j=1,J}\right]$
quantities for all such parameters. 
We give an example here, assuming that the aerosols are bimodal,
with two lognormal size distributions described by geometric standard
deviations ($\sg\fine$ and $\sg\coarse$), geometric median radii 
($\rg\fine$ and $\rg\coarse$), and non-sphericity parameters ($\epsilon\fine$ and
$\epsilon\coarse$) for the fine and coarse modes. We note
that $\epsilon$ is available only when non-spherical particles are assumed 
(T-matrix code is applied). Complex refractive indices are 
$\mreal\fine-\mimag\fine i$ and $\mreal\coarse-\mimag\coarse i$. Given
these microphysical properties, the linearized Mie/T-matrix codes will compute
for each mode the scattering and extinction efficiencies ($\qsca$ and $\qext$), 
the set of expansion coefficients ($\baer^j$) of scattering phase matrix, 
as well as the derivatives of these quantities with respect to these 
microphysical properties. For a wide size range of aerosol particles, 
which enable am about 100\% accumulated
value for the bi-lognormal probability function, the optical thickness for
aerosol extinction and scattering and the associated Greek matrix coefficients
within for one atmospheric layer can be calculated through
\begingroup
\allowdisplaybreaks
\begin{align}
\taua &= \taua\fine + \taua\coarse 
       = \frac{3V_0\fine\qext\fine}{4\reff\fine} +
         \frac{3V_0\coarse\qext\coarse}{4\reff\coarse} \label{eq:apptau} \\
\scaa &= \scaa\fine + \scaa\coarse 
       = \frac{3V_0\fine\qsca\fine}{4\reff\fine} + 
         \frac{3V_0\coarse\qsca\coarse}{4\reff\coarse} \\
\baer^j &= \frac{\scaa\fine\baer^{\text{f}j}+\scaa\coarse\baer^{\text{c}j}}
                {\scaa\fine + \scaa\coarse} \label{eq:appbaer}
\end{align}
\endgroup

We can compute vector 
$\left[ \phi_x^\prime, \varphi_x^\prime, \left<\pmb{\Psi}_x^{\prime j}\right>_{j=1,J}\right]$
for a given parameter by differentiating above equations
\eqref{eq:apptau}--\eqref{eq:appbaer}. For $x=V_0\fine$ as an example:
\begingroup
\allowdisplaybreaks
\begin{align}
\phi_x^\prime &= V_0\fine\frac{\pttaua}{\partial{V_0\fine}}
               = V_0\fine\frac{3\qext\fine}{4\reff\fine}=\taua\fine \\
\varphi_x^\prime &= V_0\fine\frac{\ptscaa}{\partial{V_0\fine}}
                  = V_0\fine\frac{3\qsca\fine}{4\reff\fine}=\scaa\fine \\
\pmb{\Psi}_x^{\prime j} 
           &= V_0\fine\frac{\ptbaer^j}{\partial{V_0\fine}} 
            = \frac{\scaa\fine}{\scaa}(\baer^{\text{f}j}-\baer^j)  
\end{align}
\endgroup

And similarly for $x=\rg\fine$, we have 
\begingroup
\allowdisplaybreaks
\begin{align}
\phi_x^\prime &= \taua\fine 
      \left(\frac{\rg\fine}{\qext\fine}\frac{\ptqext\fine}{\ptrg\fine}
           -\frac{\rg\fine}{\reff\fine}\frac{\ptreff\fine}{\ptrg\fine}\right) \\
\varphi_x^\prime &= \scaa\fine
      \left(\frac{\rg\fine}{\qsca\fine}\frac{\ptqsca\fine}{\ptrg\fine}
           -\frac{\rg\fine}{\reff\fine}\frac{\ptreff\fine}{\ptrg\fine}\right) \\
\pmb{\Psi}_x^{\prime j}  
           &= \frac{\varphi_x^\prime}{\scaa}(\baer^{\text{f},j}-\baer^j)
            + \rg\fine\frac{\ptbaer^{\text{s}j}}{\ptrg\fine}
\end{align}
\endgroup

In a similar fashion, we can obtain the vector 
$\left[ \phi_x^\prime, \varphi_x^\prime, \left<\pmb{\Psi}_x^{\prime j}\right>_{j=1,J}\right]$
for other fine-mode aerosol parameters including 
$\taua\fine$, $\assa\fine$, $V_0\fine$, $\mreal\fine$, $\mimag\fine$, 
$\rg\fine$, $\sg\fine$, and $\epsilon\fine$ (as listed in Table
\ref{tab:jacobian2}). For coarse-mode aerosol parameters, the derivations are
the same with superscript ‘s’ replaced by ‘c’.

We have implemented various aerosol-loading vertical profiles into the testbed,
including uniform, exponential-decreasing, and quasi-Gaussian profile shapes.
For the uniform profile, aerosols are assumed evenly distributed with height.
The layer AOD for the exponential-decreasing profile follows form
\begin{equation}
\int_{+\infty}^z\taua(z)\text{d}z = \tau_\text{a0}\exp\left(-\frac{z}{H}\right)
\end{equation}
where $\tau_\text{a0}$ is the columnar AOD, and $H$ is a scale height 
parameter. The quasi-Gaussian profile is derived from a generalized 
distribution function \citep{Spurr14}
\begin{equation}
\taua(z)=K\frac{\exp(-\gamma|z-z_\text{peak}|)}{[1+\exp(-\gamma|z-z_\text{peak}|)]^2}
\end{equation}
where $K$ is a constant related to $\tau_\text{a0}$, $\gamma$ is related to 
half-width constant, and $z_\text{peak}$ is the height having peak loading. 
Derivatives of layer aerosol optical thickness with respect to these profile 
parameters ($H$, $\gamma$, and $z_\text{peak}$ are also included in order to 
calculate Jacobians of Stokes vector to these parameters, and the vectors 
$\left[ \phi_x^\prime, \varphi_x^\prime, \left<\pmb{\Psi}_x^{\prime j}\right>_{j=1,J}\right]$ 
for these derivatives are also shown in Table \ref{tab:jacobian2}.
