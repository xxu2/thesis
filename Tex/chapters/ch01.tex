%%
\chapter{Introduction}

\section{Background and Motivation}

Atmospheric aerosols play a crucial role in the global climate change. They
affect earth energy budget directly by scattering and absorbing solar and
terrestrial radiation, and indirectly through altering the cloud formation,
lifetime, and radiative properties [Haywood and Boucher, 2000; Ramanathan et
al., 2001]. However, quantification of these effects in the current climate
models is fraught with uncertainties. The global average of aerosol effective
radiative forcing (ERF) were estimated to range from --0.1 to --1.9 Wm$^{-2}$ with the 
best estimate of --0.9 Wm$^{-2}$ [Boucher et al., 2013], indicating that the cooling
effects of aerosol might counteract the warming effects of 1.82$\pm$0.19 Wm$^{-2}$
caused by the increase of carbon dioxide since the industrial revolution [Myhre 
et al., 2013]. The climate effects of aerosol particles depend on their
geographical distribution, optical properties, and efficiency as cloud
condensation nuclei (CCN). Key quantities pertain to the aerosol optical and
cloud-forming properties include particle size distribution (PSD), chemical
composition, mixing state, and morphology [Boucher et al., 2013]. While the
daily aerosol optical depth (AOD) can be well measured from current satellite
and ground-based remote sensing instrumentations [e.g., Holben et al., 1998;
Kaufman et al., 2002], the accurate quantification of aerosol ERF is in no
small part hindered by our limited knowledge about the aerosol PSD and
refractive index (describing chemical composition and mixing state). 

To fully understand the role of aerosol particles in the global climate change, 
further development in observations along with retrieval algorithms for these
aerosol microphysical properties from different platforms are thus highly
needed [Mishchenko et al., 2004], and the focus of this two-part series study
is the characterization of aerosol properties from ground-based passive remote
sensing.

\subsection{Previous studies on aerosol microphysical retrievals}

\subsection{Challenges and Opportunities}

\section{Objectives}

\section{Organization}
