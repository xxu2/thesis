%%
\chapter{Introduction}

\section{Background and Motivation}

Atmospheric aerosols play a crucial role in the global climate change. They
affect earth energy budget directly by scattering and absorbing solar and
terrestrial radiation, and indirectly through altering the cloud formation,
lifetime, and radiative properties \citep{haywood00,ramanathan01}.
However, quantification of these effects in the current climate
models is fraught with uncertainties. The global average of aerosol effective
radiative forcing were estimated to range from --0.1 to --1.9 Wm$^{-2}$ with the 
best estimate of --0.9 Wm$^{-2}$ \citep{boucher13}, indicating that the cooling
effects of aerosol might counteract the warming effects of 1.82$\pm$0.19 Wm$^{-2}$
caused by the increase of carbon dioxide since the industrial revolution 
\citep{myhre13}. The climate effects of aerosol particles depend on their
geographical distribution, optical properties, and efficiency as cloud
condensation nuclei and ice nuclei. 
Key quantities pertain to the aerosol optical and
cloud-forming properties include particle size distribution (PSD), chemical
composition, mixing state, and morphology [Boucher et al., 2013]. While the
daily aerosol optical depth (AOD) can be well measured from current satellite
and ground-based remote sensing instrumentations \citep[e.g.,][]{holben98,kaufman02},
the accurate quantification of aerosol ERF is in no
small part hindered by our limited knowledge about the aerosol PSD and
refractive index (describing chemical composition and mixing state). 

To fully understand the role of aerosol particles in the global climate change, 
further development in observations along with retrieval algorithms for these
aerosol microphysical properties from different platforms are thus highly
needed \citep{Mishchenko04}, and the focus of this two-part series study
is the characterization of aerosol properties from ground-based passive remote
sensing.

\subsection{Previous studies on aerosol microphysical retrievals}

There have been continuous efforts in determining aerosol microphysical
properties from ground-based measurements of direct and/or diffuse solar
radiation since \citet{angstrom29} first suggested an empirical relationship
between the spectral dependency of extinction coefficients and the size of
aerosol particles. Over thirty years later, \citet{curcio61} inferred the aerosol
PSD from the spectral particulate extinction coefficients in the visible and
near-infrared regions. Soon with the effective numerical inversion technique
developed by \citet{Phillips62} and \citet{twomey63} specifically for error-involved
optimization, a number of studies explored the use of either spectral
attenuations or scattered radiances (in a small range of scattering angles) to
determine the aerosol PSD
\citep{Twomey67,Yamamoto69,Dave71,Grassl71,Herman71,King78}.
\citet{Shaw79} and \citet{Nakajima83} were among the first studies that have combined
optical scattering measurements with spectral extinctions to recover particle
size spectrum. \citet{Kaufman94} suggested useful information contained in
the sky radiances of larger scattering angles to retrieve the aerosol
scattering phase function and PSD. The first operational retrieval algorithm
for aerosol microphysical properties was introduced by \citet{Nakajima96},
when the multi-band automatic sun- and sky-scanning radiometer was deployed in
the AErosol RObotic NETwork, or the AERONET \citep{holben98}. All of
above mentioned methods treated aerosol particles as homogeneous spheres and
with refractive index assumed a priori, even though the refractive index can
highly impact the optical, especially the scattering characteristics \citep{hansen74}.
\citet{Tanaka82,Tanaka83} developed an inversion library
method to estimate the complex refractive index and PSD simultaneously from
measurements of scattered radiances polarized in the perpendicular and parallel
directions. Another concept for determining refractive index from both direct
and diffuse angular radiances was developed by \citet{Wendisch94} and 
\citet{Yamasoe98}, which were based on the fact that
sensitivities of scattered radiances to the PSD and those to the refractive
index are dominated on different scattering-angular regions. The current
AERONET operational inversion algorithm was developed by \citet{Dubovik00a},
which has heritage from algorithms developed by \citet{King78} and
\citet{Nakajima83,Nakajima96} but was implemented for simultaneous retrieval of
particle size distribution and complex refractive index with sophisticated
inclusion of multiple a priori constraints. \citet{Dubovik02,Dubovik06} further
implemented the spheroids in the particle shape consideration for desert dust
in the retrieval, and added fractional volume of non-spherical particles to the
inversion products.

\subsection{The AERONET measurements}

With over 400 locations around the word, most AERONET sites are equipped with
an automatic sun and sky scanning spectral radiometer, or the CIMEL-318 type
SunPhotometer, to measure direct and diffuse solar radiation in various
atmospheric window channels \citep{holben98}. 
Direct sun radiances at various atmospheric window channels from the
ultra-violet (UV) to near infrared are used to infer the spectral AODs with the
Beer-Lambert-Bouguer Law \citep{holben98,Smirnov00}. 


\subsection{Challenges and opportunities}

\section{Objectives}

\section{Organization}
